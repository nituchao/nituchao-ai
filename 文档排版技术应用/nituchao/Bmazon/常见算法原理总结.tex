%!TEX program = xelatex
%!TEX TS-program = xelatex
%!TEX encoding = UTF-8 Unicode

\documentclass[a4paper, 12pt, centering]{article}


% ===========================代码高亮===============================================
\usepackage{pythonhighlight}

% ===========================页面设置===============================================
\usepackage{microtype}
\usepackage{geometry}   %设置页边距的宏包
\geometry{left=2.5cm,right=2cm,top=2.5cm,bottom=2.5cm}  %设置 上、左、下、右 页边距

\usepackage{graphicx}  %插入图片的宏包
\usepackage{float} %设置图片浮动位置的宏包
\usepackage{subfigure} %插入多图时用子图显示的宏包
\usepackage[unicode]{hyperref} %超链接,支持unicode字符
\hypersetup{hidelinks} %隐藏超链接上的外框
\usepackage{xcolor}
\usepackage{indentfirst} %自动缩进

\usepackage{wallpaper} %background image
\addtolength{\wpYoffset}{8.0cm}
\usepackage{url}
\usepackage{fontspec,xltxtra,xunicode}
\defaultfontfeatures{Mapping=tex-text} %如果没有它,会有一些 tex 特殊字符无法正常使用,比如连字符
\usepackage{lmodern} %允许字体适配所有尺寸
%文章内中文自动换行,可以自行调节
\XeTeXlinebreaklocale'zh'
\XeTeXlinebreakskip=0pt plus 1pt minus 0.1pt

%%%% 下面的命令设置行间距与段落间距 %%%%
\linespread{1.4}
% \setlength{\parskip}{1ex}
\setlength{\parskip}{0.5\baselineskip}




% ===========================中文字体支持===============================================
% 字体需要根据自己电脑系统来设置
\usepackage{xeCJK}
%% 设置英文字体
\setmainfont{PingFangSC-Light}  
%% 设置中文字体
\setCJKmainfont{PingFangSC-Light}

\setCJKfamilyfont{zhhei}{PingFangSC-Light}
\newcommand*{\hei}{\CJKfamily{zhhei}}
\setCJKfamilyfont{zhkai}{PingFangSC-Light}
\newcommand*{\kai}{\CJKfamily{zhkai}}
\setCJKfamilyfont{enroman}{PingFangSC-Light}
\newcommand*{\mytimes}{\CJKfamily{enroman}}


% ===========================数学符号=============================================== 
\usepackage{amsmath,amssymb} 
\usepackage{bm} % $\bm{letter}$ 数学式中粗斜体字母的最佳方案 
\usepackage{calc} 
\usepackage{units} %单位宏包 

% 设置公式字体
\usepackage[mathrm=sym]{unicode-math}
% \setmathfont{FlankerGriffo}



% ===========================字号===============================================
\newcommand{\chuhao}{\fontsize{42pt}{\baselineskip}\selectfont}
\newcommand{\xiaochuhao}{\fontsize{36pt}{\baselineskip}\selectfont}
\newcommand{\yihao}{\fontsize{28pt}{\baselineskip}\selectfont}
\newcommand{\erhao}{\fontsize{21pt}{\baselineskip}\selectfont}
\newcommand{\xiaoerhao}{\fontsize{18pt}{\baselineskip}\selectfont}
\newcommand{\sanhao}{\fontsize{15.75pt}{\baselineskip}\selectfont}
\newcommand{\sihao}{\fontsize{14pt}{\baselineskip}\selectfont}
\newcommand{\xiaosihao}{\fontsize{12pt}{\baselineskip}\selectfont}
\newcommand{\wuhao}{\fontsize{10.5pt}{\baselineskip}\selectfont}
\newcommand{\xiaowuhao}{\fontsize{9pt}{\baselineskip}\selectfont}
\newcommand{\liuhao}{\fontsize{7.875pt}{\baselineskip}\selectfont}
\newcommand{\qihao}{\fontsize{5.25pt}{\baselineskip}\selectfont}



% ===========================定义摘要、关键词、章节等命令===============================================
\makeatletter
\renewcommand\abstract{%
\kai\textbf{\hei{摘要:}}
{\normalfont\xiaosihao\kai}}
\makeatother

\newenvironment{enabstract}

\makeatletter
\renewcommand\enabstract{%
\textbf{Abstract: }
{\normalfont\xiaosihao\mytimes}}
\makeatother

%%%% 设置 key words 属性 %%%%
\newenvironment{keys}

\makeatletter
\renewcommand\keys{%
\kai\textbf{\hei{关键词:}}
{\normalfont\xiaosihao\hei}}
\makeatother

\newenvironment{enkeys}

\makeatletter
\renewcommand\enkeys{%
\textbf{Key words: }
{\normalfont\xiaosihao\mytimes}}
\makeatother

%%%% 设置 section 属性 %%%%
\makeatletter
\renewcommand\section{\@startsection{section}{1}{\z@}%
{-1.5ex \@plus -.5ex \@minus -.2ex}%
{.5ex \@plus .1ex}%
{\normalfont\sihao\bf\hei}}
\makeatother

%%%% 设置 subsection 属性 %%%%
\makeatletter
\renewcommand\subsection{\@startsection{subsection}{1}{\z@}%
{-1.25ex \@plus -.5ex \@minus -.2ex}%
{.4ex \@plus .1ex}%
{\normalfont\xiaosihao\bf\hei}}
\makeatother

%%%% 设置 subsubsection 属性 %%%%
\makeatletter
\renewcommand\subsubsection{\@startsection{subsubsection}{1}{\z@}%
{-1ex \@plus -.5ex \@minus -.2ex}%
{.3ex \@plus .1ex}%
{\normalfont\xiaosihao\hei}}
\makeatother

%%%% 段落首行缩进两个字 %%%%
\makeatletter
\let\@afterindentfalse\@afterindenttrue
\@afterindenttrue
\makeatother
\setlength{\parindent}{2em}  %中文缩进两个汉字位






%%%% 定理类环境的定义 %%%%
\newtheorem{example}{例}             % 整体编号
\newtheorem{algorithm}{算法}
\newtheorem{theorem}{定理}[section]  % 按 section 编号
\newtheorem{definition}{定义}
\newtheorem{axiom}{公理}
\newtheorem{property}{性质}
\newtheorem{proposition}{命题}
\newtheorem{lemma}{引理}
\newtheorem{corollary}{推论}
\newtheorem{remark}{注解}
\newtheorem{condition}{条件}
\newtheorem{conclusion}{结论}
\newtheorem{assumption}{假设}

%%%% 重定义 %%%%
\renewcommand{\contentsname}{目录}  % 将Contents改为目录
\renewcommand{\abstractname}{摘要}  % 将Abstract改为摘要
\renewcommand{\refname}{参考文献}   % 将References改为参考文献
\renewcommand{\indexname}{索引}
\renewcommand{\figurename}{图}
\renewcommand{\tablename}{表}
\renewcommand{\appendixname}{附录}
\renewcommand{\algorithm}{算法}


%%%% 定义标题格式,包括title,author,affiliation,email等 %%%%
\title{\yihao{工作中高优算法总结}}
\author{\xiaoerhao{zhangliang605}\footnote{电子邮件: zhangliang605@gmail.com}\\[2ex]
\sanhao{} DATOUBANG.INC\\[2ex]
}
\date{\sanhao\today}




% ===========================正正正正正正正=======================================
% ===========================文文文文文文文=======================================
% ===========================这这这这这这这=======================================
% ===========================里里里里里里里=======================================
% ===========================开开开开开开开=======================================
% ===========================始始始始始始始=======================================
% ===========================写写写写写写写=======================================
% ===========================哟哟哟哟哟哟哟=======================================



\begin{document}

%设置公式行间距为1.5倍
\linespread{1.5}

%封面背景图片
%可以换其它自己想放的图片
\ThisCenterWallPaper{1.0}{img/banner.jpg}






%%%% 以下部分是正文 %%%%  

%标题上方空出一些距离
\
\vspace{10cm}
\begingroup
\let\newpage\relax% Void the actions of \newpage
\maketitle
\endgroup
\newpage
%----------


%%% 摘要 %%%

\par 这个文档是我工作中遇到的各种算法的总结,本意是希望能对自己的知识体系进行系统性的总结归纳。
随着工作时间越来越长,渐渐地,我有一种比较强烈的感觉,所谓"吾生也有涯,而知无涯"。一味地追求知识
固然是一件正确的事情,但学习的同时也要特别关注创造。一个知识点会被很多人学习,但是创造往往是少数
人能做到的事情。
\par 因此,这个文档也寄托了我希望能有些创造性的产品或者想法产生的期望。
\par 

\par 关键字:整数规划,背包问题,营销出价,广告定价,流量控制,数据结构算法
\newpage

%%% 目录 %%%
\tableofcontents
\newpage

%%%%%%%%%%%%%%%%%%%%%%%%%%%%%%%%%%%%%%%%%%%%%%%%%%%%%%%%%%%%%%%%%%%%%%%%%%%%%%%%
%% PID算法字啊成本控制领域的应用
%%%%%%%%%%%%%%%%%%%%%%%%%%%%%%%%%%%%%%%%%%%%%%%%%%%%%%%%%%%%%%%%%%%%%%%%%%%%%%%%
\section{PID算法在成本控制领域的应用}
常见的成本控制方法可分为认为干预和算法自动控制两种。顾名思义,人为干预是通过人工实时监控广告投放
情况,当发现实际成本低于或超出预期预算时,通过人工调整广告出价或修改人群定向等方式调节投放花费;
算法自动控制是指采用相关算法,监控投放成本,并根据异常自动调节广告出价,达到控制成本的目的。
\subsection{PID控制算法简介}
PID算法包含了比例(Proportion)、积分(Integration)、微分(Differentiation)三个环节,
其根据被控对象实际输出与目标值的偏差,按照三个环节进行运算,最终达到稳定系统的目的。\\

简答的说,$k_{p}$代表现在,$k_{i}$代表过去,$k_{d}$代表未来。在实际应用中还是需要考虑具体参
数大小,可以通过grid search,根据相应时间、超调量、稳态误差指标,来综合考虑PID值。\\

PID调价也存在着一些缺陷,简单泛化能力强式优点也是缺点,只需要根据设定$cpc$和实际$cpc$的反馈就
能够调节。但是,在某些固定场景下,$cpc$的波动会呈现固定的pattern,例如在某几个小时流量指令非常
好,$cpc$会特别低,这就需要使用机器学习来记忆到哪些campaingn在哪些时间点需要提高价格,使用强化
学习出价在充分利用投放数据、建立MDP模型、序列号决策这些方面就有了天然优势。\\

PID具体公式如下:
\begin{equation}
    \begin{split}
        \displaystyle err_{t} &= target_{cpc_{t}} - real_{cpc_{t}}\\
        \displaystyle \Delta_{t} &= k_{p}(err(t) + \frac{1}{k_{i}} \int{err(t)\mathrm{d}t } +k_{d}\frac{\mathrm{d} err(x)}{\mathrm{d} t} ) \\
        \displaystyle \lambda{_{t+1}} &= \lambda_{t} + \Delta_{t} 
    \end{split}
\end{equation}

其中:
\begin{itemize}
    \item $err\_{t}$: 第t轮PID的误差值
    \item $real\_{cpc_{t}}$: 第t轮PID的实际值
    \item $target\_{cpc_{t}}$: 第t轮PID的目标值
    \item $k\_{p}$: 比例增益
    \item $k\_{i}$: 积分时间常数
    \item $k\_{d}$: 微分时间常数
    \item $\Delta$: 第t轮PID的增量系数
    \item $\lambda_{t}$: 第t轮PID的调控系数
    \item $\lambda_{t-1}$: 第t-1轮PID的调控系数
\end{itemize}

\paragraph{关于P}
$k_{p}$是比例系数,假设目标cpc 0.4,实际cpc 0.2,误差是0.2,$k_{p}$越大,反应幅度就会越大,
新的$\lambda$就会增加很多,出价就会增加很多,但是$k_{p}$不能够过大,不然会导致超额调整,出价过
高。所以,$k_{p}$代表了根据当前误差反应的比例。

\paragraph{关于I}
$k_{p}$的存在是为了解决稳态误差。\\
假如当前cpc偏低,每个小时都提高价格,但是市场价格(出价第二高的广告主出价)也在下降,所以,虽然每
个小时都按照PID调控系数提价,但是由于市场价格在降价,导致基于PID的每次提价都没有提上去。像这种如
果一直存在,我们称之为稳态误差。积分的存在就是通过过去差值的经验来调整出价,来消除这个稳态误差。

但是,实际投放过程中基本不会存在这样的稳态,因为竞价系统是动态的,只能说市场价格可能随着时间有些固
定的变化,但是变化不一定式稳定方向,所以$k_{i}$值在实际使用中需要慎重,如果设置特别大,会导致上个
小时已经不存在的误差,影响到当前小时,所以$k_{i}$即使要使用,最好设置的非常小。

\paragraph{关于D}
$k_{d}$项经常被称为微分项,当两次调控间隔十分小,$(err_{t} - err_{t-1})/1$计算的就是斜率,
如果间隔十分小,那么这个斜率就可以一定程度体现次$err$的走向,这也是为社么说微分项代表未来。但是,
如果两次间隔十分大、或者噪音非常多,微分项的作用就不大了。对于1小时调控一次的PID调价,$k_{d}$项
可以为0。实际上,很多PID控制器仅用$k_{p}$和$k_{i}$就已经足够了。

\subsection{PID算法简单代码}
PID控制算法可以分为位置时PID和增量式PID控制算法。
两者的区别:
\begin{itemize}
    \item (1)位置式PID控制的输出与整个过去的状态有关,用到了误差的累加值。而增量式PID的输出
    只与当前拍和前两拍的误差有关,因此位置式PID控制的累计误差相对更大。
    \item (2)增量式PID控制输出的是控制量增量,如果计算机出现故障,误动作影响较小,而执行机构
    本身有记忆功能,仍可保持原位,不会严重影响系统的工作,而位置式的输出直接对应对象的输出,因此
    对系统影响较大。
\end{itemize}

\subsubsection{位置式PID}
\begin{equation}
    \displaystyle u(k) = K_{P}e(k) + K_{I}\sum_{i=0}e(i) + K_{D}[e(k) - e(k-1)]
\end{equation}
%% Python代码
\begin{python}
# 位置型PID
def pid_positional(budget_target, budget_error):
    P = budget_error[0]
    I = sum(Budget_error)
    D = budget_error[0] - Budget_error[1]

    return kp * P + ki * I + kd * D
\end{python}

\subsubsection{增量式PID}
%% 数学公式
\begin{equation}
    \begin{split}
        \displaystyle \Delta u(k) &= u(k) - u(k-1)\vspace{1.5ex} \\
        \displaystyle &= K_{p}[e(k) - e(k-1)] + K_{I}e(k) + K_{D}[e(k) - 2e(k-1) + e(k-2)]
    \end{split}
\end{equation}
%% Python代码
\begin{python}
import pandas as pd
import random
import matplotlib.pyplot as plt

# 模拟调控N天,第1天无法用PID,从第2天开始施加PID调控
N = 10

# 每天目标预算
Budget_target = 100

# 每天实际花出预算
Budget_real = [0] * N                       # 历次PID调控给出的真实预算
Budget_real[1] = 80

# 调控后每天预期预算
Budget_expected_positional = [0] * N
Budget_expected_incremental = [0] * N       # 历次PID调控后预期预算
Budget_expected_incremental_u = [0] * N     # 历次PID调控后预算差值 = 预期预算 - 目标预算

# PID系数
kp = 0.5
ki = 0.5
kd = 0.5

# 积分天数
T = 3

# 增量型PID
def pid_incremental(Budget_target, Budget_error, u):

  delta = kp * (Budget_error[0] - Budget_error[1]) + ki * Budget_error[0] + kd * (Budget_error[0] - 2 * Budget_error[1] + Budget_error[2])

  return delta + u

# 模拟每天实际花出预算,在目标预算和预期预算之间
def get_budget(Budget_min, Budget_max):
  return random.uniform(Budget_min, Budget_max)

# 第2天开始PID调控
for i in range(2, N):
  Budget_error = []
  for j in range(i-1, max(0, i-T), -1):
    Budget_error.append(Budget_target - Budget_real[j])
  if len(Budget_error) < T:
    Budget_error = Budget_error + [0] * (T - len(Budget_error))

  # 增量型PID
  Budget_expected_incremental_u[i] = pid_incremental(Budget_target, Budget_error, Budget_expected_incremental_u[i-1])
  Budget_expected_incremental[i] = Budget_target + Budget_expected_incremental_u[i]
  Budget_min = min(Budget_target, Budget_expected_incremental[i])
  Budget_max = max(Budget_target, Budget_expected_incremental[i])
  Budget_real[i] = get_budget(Budget_min, Budget_max)
\end{python}

\subsection{参考资料}
\begin{itemize}
    \item \href{https://www.infoq.cn/article/akkwpvsnium9tmhhuu3f}{PID 算法在广告成本控制领域的应用}
    \item \href{https://www.zhihu.com/tardis/zm/art/139244173?source_id=1003}{广告出价--如何使用PID控制广告投放成本}
\end{itemize}
\newpage

%%%%%%%%%%%%%%%%%%%%%%%%%%%%%%%%%%%%%%%%%%%%%%%%%%%%%%%%%%%%%%%%%%%%%%%%%%%%%%%%
%% Uplift算法在成本定价领域的应用
%%%%%%%%%%%%%%%%%%%%%%%%%%%%%%%%%%%%%%%%%%%%%%%%%%%%%%%%%%%%%%%%%%%%%%%%%%%%%%%%
\section{Uplift算法在成本定价领域的应用}
\newpage

%%%%%%%%%%%%%%%%%%%%%%%%%%%%%%%%%%%%%%%%%%%%%%%%%%%%%%%%%%%%%%%%%%%%%%%%%%%%%%%%
%% 因果推断建模
%%%%%%%%%%%%%%%%%%%%%%%%%%%%%%%%%%%%%%%%%%%%%%%%%%%%%%%%%%%%%%%%%%%%%%%%%%%%%%%%
\section{因果推断建模}
\subsection{参考资料}
\begin{itemize}
    \item \href{https://tech.meituan.com/2024/01/25/identify-causal-effect.html}
    {分布式因果推断在美团履约平台的探索和实践}
    \item \href{https://www.modb.pro/db/601027}{因果推断之Uplift Model|CausalML实战篇}
    \item \href{https://github.com/uber/causalml}{CausalML: A Python Package for
    Uplift Modeling and Causal Inference with ML}
    \item \href{https://causalml.readthedocs.io/en/latest/about.html}{About 
    CausalML}
\end{itemize}
\newpage

%%%%%%%%%%%%%%%%%%%%%%%%%%%%%%%%%%%%%%%%%%%%%%%%%%%%%%%%%%%%%%%%%%%%%%%%%%%%%%%%
%% 多约束整数规划
%%%%%%%%%%%%%%%%%%%%%%%%%%%%%%%%%%%%%%%%%%%%%%%%%%%%%%%%%%%%%%%%%%%%%%%%%%%%%%%%
\section{多约束整数规划}
\subsection{贪婪算法}
\subsection{二分算法}
\subsection{单纯形法}
\subsection{分枝界定法}
\subsection{启发式算法}
\subsection{拉格朗日乘子法}
\subsection{同步坐标下降法}
\newpage

%%%%%%%%%%%%%%%%%%%%%%%%%%%%%%%%%%%%%%%%%%%%%%%%%%%%%%%%%%%%%%%%%%%%%%%%%%%%%%%%
%% ROI公式推导
%%%%%%%%%%%%%%%%%%%%%%%%%%%%%%%%%%%%%%%%%%%%%%%%%%%%%%%%%%%%%%%%%%%%%%%%%%%%%%%%
\section{ROI公式推导}
\subsection{ROI公式}
\begin{align*}
    \displaystyle  ROI = \frac{CVR^{30} - CVR^{5}}{CVR^{30} * 30 - CVR^{5} * 5}  \propto \frac{CVR^{30} - CVR^{5}}{CVR^{5}}
\end{align*}
\subsection{CVR评分选择}
假定选择两个价格档位,最低档5元,最高档30元(选择最高档和最低档,价格敏感性较为明显,容易学出来)。
\subsection{边际ROI公式推导}
边际ROI公式推导
\begin{equation}
    \begin{split}
        \displaystyle \mbox{边际}ROI &= \frac{bk^{30} - bk^{5}}{cost^{30} - cost^{5}}\vspace{1.5ex} \\
        \displaystyle               &= \frac{bk^{30} - bk^{5}}{bk^{30} * 30 - bk^{5} * 5}\vspace{1.5ex} \\
        \displaystyle               &= \frac{CVR^{30} - CVR^{5}}{CVR^{30} * 30 - CVR^{5} * 5}\vspace{1.5ex}
    \end{split}
\end{equation}
分子分母同时除以曝光量UV,假设5元档和30元档的曝光量UV是拉齐的,(如果不拉齐,就需要归一操作)
\subsection{正比公式推导}
正比公式推导
\begin{equation}
    \begin{split}
        \displaystyle \mbox{边际}CAC &= \frac{1}{\mbox{边际}ROI}\vspace{1.5ex} \\
        \displaystyle               &= \frac{CVR^{30} * 30 - CVR^{5} * 5}{CVR^{30} - CVR^{5}}\vspace{1.5ex} \\
        \displaystyle               &= \frac{CVR^{30} * 30 - CVR^{5} * 30 + CVR^{5} * 30 - CVR^{5} * 5}{CVR^{30} - CVR^{5}}\vspace{1.5ex}  \\
        \displaystyle               &= 30 + \frac{CVR^{5} * 25}{CVR^{30} -CVR^{5}}\vspace{1.5ex} \\
        \displaystyle               &= 30 + \frac{CVR^5}{CVR^{30} - CVR^{5}} * 25\vspace{1.5ex}
    \end{split}
\end{equation}
因此,$\displaystyle \mbox{边际}CAC \propto \frac{CVR^{5}}{CVR^{30} - CVR^{5}}$
因此,$\displaystyle \mbox{边际}ROI \propto \frac{CVR^{5}}{CVR^{30} - CVR^{5}}$
\newpage

\begin{align*}
    &\max \sum_{i,j}x_{i,j}\cdot bind\_card_{i,j} \\
    &s.t.\quad
    \sum_{i,j=1}x_{i,j}\cdot (all\_ctr_{i,j} - bytepay\_ctr_{i,j}) <= \beta \cdot \left | PV \right | \cdot all\_ctr_{emp} \\
    \sum_{j}x_{i,j} = 1, x_{i,j} \in {0, 1}
\end{align*}

\begin{align*}
    &\max \sum_{i,j}x_{i,j}\cdot bind\_card_{i,j} \\
    &s.t.& \sum_{i,j=1}x_{i,j}\cdot (all\_ctr_{i,j} - bytepay\_ctr_{i,j}) <= \beta \cdot \left | PV \right | \cdot all\_ctr_{emp} \\
        & \sum_{j}x_{i,j} = 1, x_{i,j} \in {0, 1}
\end{align*}

\begin{thebibliography}{1}
\bibitem{1} Wray J, Green G G R. Neural networks, approximation theory, and 
finite precision computation[J]. Neural networks, 1995, 8(1): 31-37.
\bibitem{2} Ham F M, Kostanic I. Principles of neurocomputing for science and 
engineering[M]. McGraw-Hill Higher Education, 2000.
\end{thebibliography}

\end{document}